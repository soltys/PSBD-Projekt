\documentclass[a4paper]{article}
\usepackage{placeins}
\usepackage{multicol}
\usepackage{parskip}

\usepackage{tabu}

\usepackage{polski}
\usepackage[utf8]{inputenc} 
\usepackage{amsmath}
\usepackage[pdftex]{graphicx}

\newcommand{\HRule}{\rule{\linewidth}{0.5mm}}

\title{Programowanie serwerów baz danych}
\author{Paweł Sołtysiak, Marcin Klepacki}
\date{\today}

\begin{document}
\begin{titlepage}
\begin{center}

% Upper part of the page. The '~' is needed because \\
% only works if a paragraph has started.

\begin{figure}[ht]
\begin{minipage}[b]{0.45\linewidth}
\centering
\includegraphics[width=\textwidth]{tex/zut_logo}

\end{minipage}
\hspace{0.5cm}
\begin{minipage}[b]{0.45\linewidth}
\centering
\includegraphics[width=\textwidth,height=2cm]{tex/wi_logo}

\end{minipage}
\end{figure}

\textsc{\LARGE Programowanie serwerów baz danych}\\[1.5cm]

\textsc{\Large Projekt}\\[0.5cm]

% Title
\HRule \\[0.4cm]
{ \huge \bfseries System bazodanowy\\ zarządzanie zasobami ludzkimi}\\[0.4cm]

\HRule \\[1.5cm]


\begin{minipage}{0.4\textwidth}
\begin{flushleft} \large
\emph{Autorzy:}\\
Paweł \textsc{Sołtysiak}
\end{flushleft}
\end{minipage}
\begin{minipage}{0.4\textwidth}
\begin{flushright} \large
%\emph{Supervisor:} 
~\\
Marcin \textsc{Klepacki}
\end{flushright}
\end{minipage}
~\\[0.5cm]
\begin{minipage}{0.82\textwidth}
\begin{flushleft} \large
\emph{Grupa:}\\
\textsc{I1-32B}
\end{flushleft}
\end{minipage}


\vfill

% Bottom of the page
{\large \today}

\end{center}
\end{titlepage}

\section{Cel projektu}
Celem projektu jest opracowanie modelu danych oraz mechanizmów programowania baz danych dla systemu wspomagającego zarządzanie zasobami ludzkimi.
\section{Zakres funkcjonalny}
System ten będzie przechowywał podstawowe informacje, które usprawnią działanie działu kadr w firmie IT. Baza będzie gromadzić informacje o pracownikach, ich kompetencjach, szkoleniach itp.
\subsection{Moduły funkcjonalne}
\begin{itemize}
\item Moduł \textbf{Pracownik} -- ewidencja pracowników
\item Moduł \textbf{Szkolenia} -- Lista dostępnych szkoleń, możliwość zapisania pracownika na szkolenie.
\item Moduł \textbf{Stanowisko} -- Hierarchia stanowisk w firmie
\item Moduł \textbf{Administracja pracownikami} -- Przyznawanie urlopów, odnotowywanie zwolnień lekarskich.
\end{itemize}
\subsection{Kategorie użytkowników systemu i ich uprawnienia (realizowane funkcje)}
\begin{itemize}
\item \textbf{Administrator} -- pracownik działu kadr, posiada pełen dostęp do bazy danych.
\item \textbf{Pracownik} -- możliwość zgłaszania się na szkolenie, oraz składania podania o urlop.

\end{itemize}
\section{Model semantyczny danych}
\subsection{Obiekty rzeczywiste i abstrakcyjne systemu}
\begin{multicols}{2}
\begin{enumerate}


\item Pracownik
\item Dane-osobowe
\item Państwo
\item Miasto
\item Umiejętność
\item Poziom-Umiejętności
\item Typ-Zatrudnienia
\item Historia-Zatrudnienia
\item Firma
\item Uprawnienie
\item Szkolenie
\item Szkolenie-Planowane
\item Stanowisko
\item Zwolnienie-Lekarskie
\item Urlop
\item Kandydat

\end{enumerate}
\end{multicols}
\subsection{Powiązanie między obiektami}
Spis relacji występujących w systemie bazodanowym


\begin{table}[!htbp]
  \centering
  \begin{tabu}{|l|l|c|c|}
  \hline
	\large\textbf{Rodzic}&\large\textbf{Dziecko} &\large\textbf{Relacja}&\large\textbf{FK / PK}\\
	
	%PRACOWNIK
	\hline
	Dane Osobowe & Pracownik  & $1:1$ & \texttt{PK} \\	
	\hline		
	Umiejętność & Pracownik  & $M:N$ &  --- \\
	\hline	
	Typ-Zatrudnienia & Pracownik & $0:1$ & \texttt{FK}\\
	\hline
	Historia-Zatrudnienia & Pracownik & $0:*$  & \texttt{PK}\\
	\hline
	Uprawnienie & Pracownik & $M:N$ & ---\\
	\hline
	Szkolenie & Pracownik & $0:*$ & \texttt{PK}\\
	\hline
	Szkolenie-Planowane & Pracownik & $0:*$ & \texttt{PK}\\
	\hline
	Stanowisko & Pracownik& $1:*$ &	\texttt{FK}\\
	\hline
	Zwolnienie-Lekarskie & Pracownik & $0:*$ & \texttt{PK}\\
	\hline
	Urlop & Pracownik & $0:*$ & \texttt{PK}\\	
	
	%DANE OSOBOWE
	\hline	
	Miasto &  Dane-Osobowe &$1:*$ &\texttt{FK}\\
	
	%PAŃSTWO
	\hline
	Państwo & Miasto & $1:*$ & \texttt{FK}	\\
	
	%UMIEJĘTNOŚCI
	\hline
	Poziom-Umiejętności & Umiejętność & $1:*$ & \texttt{FK}\\	

	%FIRMA
	\hline	
	Firma & Historia-Zatrudnienia & $0:*$& \texttt{FK}\\	
	
	%Ostatnia linia
	\hline
  \end{tabu}
  \caption{Powiązania pomiędzy obiektami}
  \label{tab:myfirsttable}
\end{table}


\end{document}


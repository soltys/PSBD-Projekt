\documentclass[a4paper]{article}

\usepackage{placeins}
\usepackage{multicol}
\usepackage{parskip}
\frenchspacing
\usepackage{indentfirst}
\usepackage{hyperref}
\hypersetup{
    colorlinks,
    citecolor=black,
    filecolor=black,
    linkcolor=black,
    urlcolor=black
}
\usepackage{tabu}
\usepackage{longtable}
\usepackage{polski}
\usepackage[utf8]{inputenc} 
\usepackage{amsmath}
\usepackage[pdftex]{graphicx}

\newcommand{\HRule}{\rule{\linewidth}{0.5mm}}

\usepackage{custom-tables}

\title{Programowanie serwerów baz danych}
\author{Paweł Sołtysiak, Marcin Klepacki}
\date{\today}

\begin{document}
\begin{titlepage}
\begin{center}

% Upper part of the page. The '~' is needed because \\
% only works if a paragraph has started.

\begin{figure}[ht]
\begin{minipage}[b]{0.45\linewidth}
\centering
\includegraphics[width=\textwidth]{tex/zut_logo}

\end{minipage}
\hspace{0.5cm}
\begin{minipage}[b]{0.45\linewidth}
\centering
\includegraphics[width=\textwidth,height=2cm]{tex/wi_logo}

\end{minipage}
\end{figure}

\textsc{\LARGE Programowanie serwerów baz danych}\\[1.5cm]

\textsc{\Large Projekt}\\[0.5cm]

% Title
\HRule \\[0.4cm]
{ \huge \bfseries System bazodanowy\\ zarządzanie zasobami ludzkimi}\\[0.4cm]

\HRule \\[1.5cm]


\begin{minipage}{0.4\textwidth}
\begin{flushleft} \large
\emph{Autorzy:}\\
Paweł \textsc{Sołtysiak}
\end{flushleft}
\end{minipage}
\begin{minipage}{0.4\textwidth}
\begin{flushright} \large
%\emph{Supervisor:} 
~\\
Marcin \textsc{Klepacki}
\end{flushright}
\end{minipage}
~\\[0.5cm]
\begin{minipage}{0.82\textwidth}
\begin{flushleft} \large
\emph{Grupa:}\\
\textsc{I1-32B}
\end{flushleft}
\end{minipage}


\vfill

% Bottom of the page
{\large \today}

\end{center}
\end{titlepage}
\tableofcontents
\section{Cel projektu}
Celem projektu jest opracowanie modelu danych oraz mechanizmów programowania baz danych dla systemu wspomagającego zarządzanie zasobami ludzkimi.

\section{Zakres funkcjonalny}
System ten będzie przechowywał podstawowe informacje, które usprawnią działanie działu kadr w firmie IT. Baza będzie gromadzić informacje o pracownikach, ich kompetencjach, szkoleniach itp.
\subsection{Moduły funkcjonalne}
\begin{itemize}
\item Moduł \textbf{Osoba} -- ewidencja pracowników oraz kandydatów.
\item Moduł \textbf{Szkolenia} -- Lista dostępnych szkoleń, możliwość zapisania pracownika na szkolenie.
\item Moduł \textbf{Stanowisko} -- Hierarchia stanowisk w firmie.
\item Moduł \textbf{Administracja pracownikami} -- Przyznawanie urlopów, odnotowywanie zwolnień lekarskich.
\end{itemize}
\subsection{Kategorie użytkowników systemu i ich uprawnienia (realizowane funkcje)}
\begin{itemize}
\item \textbf{Administrator} -- pracownik działu kadr, posiada pełen dostęp do bazy danych.
\item \textbf{Pracownik} -- możliwość zgłaszania się na szkolenie, oraz składania podania o urlop.

\end{itemize}
\section{Model semantyczny danych}
\subsection{Obiekty rzeczywiste i abstrakcyjne systemu}
\begin{multicols}{2}
\begin{enumerate}


\item Osoba
\item Państwo
\item Miasto
\item Umiejętność
\item Poziom-Umiejętności
\item Typ-Zatrudnienia
\item Historia-Zatrudnienia
\item Firma
\item Uprawnienie
\item Certyfikat
\item Szkolenie
\item Typ-Szkolenia
\item Stanowisko
\item Oddział
\item Dział
\item Urlop
\item Typ-Urlop
\item Umowa
\item Typ-Umowa
\item Stan-cywilny
\item Język-obcy
\item Rodzaj-Znajomości
\item Historia-Edukacji
\item Szkoła
\item Hobby
\end{enumerate}
\end{multicols}
\subsection{Powiązanie między obiektami}
Spis relacji występujących w systemie bazodanowym, znajduje się w tabeli nr \ref{tab:database-relations} na stronie \pageref{tab:database-relations}, przy wykorzystaniu notacji (\emph{min-max})

\begin{relationsTable}{Spis relacji}{tab:database-relations}

%PRACOWNIK
	\hline	
	Miasto & Pracownik &$1:*$ &\texttt{FK}\\	
	\hline		
	Umiejętność & Pracownik  & $M:N$ &  --- \\
	\hline
	Historia-Zatrudnienia & Pracownik & $M:N$  & ---\\
	\hline
	Uprawnienie & Pracownik & $M:N$ & ---\\
	\hline
	Certyfikat & Pracownik & $M:N$ & ---\\
	\hline
	Szkolenie & Pracownik & $M:N$ & ---\\
	\hline
	Stanowisko & Pracownik& $0:*$ &	\texttt{FK}\\
	\hline
	Odział & Pracownik& $0:*$ &	\texttt{FK}\\
	\hline
	Pracownik  & Urlop  & $0:*$ & \texttt{PK}\\	
	\hline
	Umowa &Pracownik & $0:1$ & \texttt{PK}\\
	\hline
	Stan-cywilny & Pracownik & $1:*$ & \texttt{FK}\\
	\hline
	Język-obcy & Pracownik & $M:N$ & ---\\
	\hline
	Historia-Edukacji & Pracownik & $M:N$ & --- \\
	\hline
	Hobby & Pracownik & $M:N$ & ---\\
	%PAŃSTWO
	\hline
	Państwo & Miasto & $1:*$ & \texttt{FK}	\\
	
	%MIASTO
	\hline
	Miasto & Firma & $1:*$ & \texttt{FK}\\
	\hline
	Miasto & Szkoła & $1:*$ & \texttt{FK}\\
		
	%UMIEJĘTNOŚĆ
	\hline
	Poziom-Umiejętności &	Umiejętność & $1:*$ & \texttt{FK}\\
	\hline
    Umiejętność &  Szkolenie & $1:*$ & \texttt{FK}\\
	
	%Historia-Zatrudnienia
	\hline
	Typ-Zatrudnienia & Historia-Zatrudnienia & $1:*$ & \texttt{FK}\\
	\hline
	Firma & Historia-Zatrudnienia & $1:*$ & \texttt{FK}\\
	
	%SZKOLENIE
	\hline
	Typ-Szkolenia & Szkolenie & $1:*$ & \texttt{FK}\\
	
	%Odział
	\hline
	Dział &	Odział & $1:*$ &\texttt{FK}\\
	
	%URLOP
	\hline
	Typ-Urlop & Urlop & $1:*$ & \texttt{FK}\\
	
	%UMOWA
	\hline
	Typ-Umowa & Umowa & $1:*$ & \texttt{FK}\\
	
	%JĘZYK OBCY
	\hline
	Rodzaj-Znajomość & Język-obcy & $1:*$ & \texttt{FK}\\
	
	%Historia-Edukacji
	\hline
	Szkoła & Historia-Edukacji & $1:*$ & \texttt{FK}\\
	

\end{relationsTable}


Spis relacji dla tabel skrzyżowań, znajdują się w tabeli nr \ref{tab:database-crossover} na stronie \pageref{tab:database-crossover}, przy wykorzystaniu notacji (\emph{min-max})

\begin{relationsTable}{Tabele skrzyżowań}{tab:database-crossover}  

 \hline	
	 Pracownik & Lista-Umiejętności & $0:*$ &\texttt{PK}\\	
	 	\hline	
	 Umiejętność & Lista-Umiejętności & $0:*$ &\texttt{PK}\\
	\hline	
	Pracownik & Lista-Historii-Zatrudnienia & $0:*$ &\texttt{PK}\\	
	\hline	
	Historia-Zatrudnienia & Lista-Historii-Zatrudnienia & $0:*$ &\texttt{PK}\\	
	\hline	
	Pracownik &Lista-Uprawnień & $0:*$ &\texttt{PK}\\	
	\hline	
	Uprawnienie &Lista-Uprawnień & $0:*$ &\texttt{PK}\\
	\hline	
	Pracownik &Lista-Certyfikatów & $0:*$ &\texttt{PK}\\	
	\hline	
	Certyfikat &Lista-Certyfikatów & $0:*$ &\texttt{PK}\\	
	\hline	
	Pracownik &Lista-Szkoleń & $0:*$ &\texttt{PK}\\	
	\hline	
	Szkolenie &Lista-Szkoleń & $0:*$ &\texttt{PK}\\	
	\hline	
	Pracownik &Lista-Języków-Obcych & $0:*$ &\texttt{PK}\\	
	\hline	
	Język-Obcy &Lista-Języków-Obcych & $0:*$ &\texttt{PK}\\	
	\hline	
	 Pracownik &Lista-Historii-Edukacji &$0:*$ &\texttt{PK}\\	
	 \hline	
	 Historia-Edukacji &Lista-Historii-Edukacji &$0:*$ &\texttt{PK}\\	
	\hline	
	Pracownik &Lista-Hobby & $0:*$ &\texttt{PK}\\	
	\hline	
	Hobby &Lista-Hobby & $0:*$ &\texttt{PK}\\	
\end{relationsTable}

\subsection{Klucze główne i pozostałe atrybuty}

\begin{attributesTable}{Osoba}
\hline
\underline{Id} & int \\
\hline
Imię & text \\
\hline
Nazwisko & text\\
\hline
Data-Urodzenia & datetime \\
\hline
Miasto-Id & int \\
\hline
Telefon & text\\
\hline
E--mail & text \\
\hline
PESEL & text \\
\hline
Stanowisko-Id & int \\
\hline
Odział-Id & int \\
\hline
Umowa-Id & int \\
\hline
Stan-Cywilny-Id & int \\
\hline
Kandydat & bool \\

%Ostatnia linia
\hline

\end{attributesTable}


\begin{attributesTable}{Państwo}
\hline
\underline{Id} & int \\
\hline
Nazwa & text \\
\end{attributesTable}

\begin{attributesTable}{Miasto}
\hline
\underline{Id} & int \\
\hline
Nazwa & text \\
\hline
Państwo-Id & int \\
\end{attributesTable}


\begin{attributesTable}{Umiejętność}
\hline
\underline{Id} & int \\
\hline
Nazwa & text \\
\hline
Poziom-Umiejętności-Id & int \\
\end{attributesTable}

\begin{attributesTable}{Poziom-Umiejętności}
\hline
\underline{Id} & int \\
\hline
Opis & text \\
\end{attributesTable}

\begin{attributesTable}{Typ-Zatrudnienia}
\hline
\underline{Id} & int \\
\hline
Nazwa & text \\
\end{attributesTable}

\begin{attributesTable}{Historia-Zatrudnienia}
\hline
\underline{Id} & int \\
\hline
Od-kiedy & datetime \\
\hline
Do-kiedy & datetime \\
\hline
Typ-Zatrudnienia-Id & int \\
\hline
Firma-Id & int \\
\end{attributesTable}


\begin{attributesTable}{Firma}
\hline
\underline{Id} & int \\
\hline
Nazwa & text \\
\hline
Miasto-Id & int\\
\end{attributesTable}


\begin{attributesTable}{Uprawnienie}
\hline
\underline{Id} & int \\
\hline
Nazwa & text \\
\hline
Ważność & datetime\\
\end{attributesTable}


\begin{attributesTable}{Certyfikat}
\hline
\underline{Id} & int \\
\hline
Nazwa & text \\
\hline
Osoba-Wystawiająca & text\\
\end{attributesTable}

\begin{attributesTable}{Szkolenie}
\hline
\underline{Id} & int \\
\hline
Nazwa & text \\
\hline
Od-kiedy & datetime\\
\hline
Do-kiedy & datetime\\
\hline
Typ-Szkolenia-Id & int \\
\end{attributesTable}

\begin{attributesTable}{Typ-Szkolenia}
\hline
\underline{Id} & int \\
\hline
Nazwa & text \\
\hline
Poziom & text\\
\end{attributesTable}


\begin{attributesTable}{Stanowisko}
\hline
\underline{Id} & int \\
\hline
Nazwa & text \\
\end{attributesTable}


\begin{attributesTable}{Oddział}
\hline
\underline{Id} & int \\
\hline
Nazwa & text \\
\hline
Dział-Id & int \\
\end{attributesTable}


\begin{attributesTable}{Dział}
\hline
\underline{Id} & int \\
\hline
Nazwa & text \\
\end{attributesTable}

\begin{attributesTable}{Urlop}

\underline{Id} & int \\
\hline
Od-kiedy & datetime\\
\hline
Do-kiedy & datetime\\
\hline
Pracownik-Id & int \\
\hline
Urlop-Typ-Id & int \\
\end{attributesTable}

\begin{attributesTable}{Typ-Urlop}
\hline
\underline{Id} & int \\
\hline
Nazwa & text \\
\end{attributesTable}


\begin{attributesTable}{Umowa}
\hline
\underline{Id} & int \\
\hline
Od-kiedy & datetime\\
\hline
Do-kiedy & datetime\\
\hline
Umowa-Typ & int \\
\end{attributesTable}

\begin{attributesTable}{Typ-Umowa}
\hline
\underline{Id} & int \\
\hline
Nazwa & text \\
\end{attributesTable}

\begin{attributesTable}{Stan-Cywilny}
\hline
\underline{Id} & int \\
\hline
Nazwa & text \\
\end{attributesTable}

\begin{attributesTable}{Język-Obcy}
\hline
\underline{Id} & int \\
\hline
Nazwa & text \\
\hline
Rodzaj-Znajomości-Id & int \\
\end{attributesTable}

\begin{attributesTable}{Rodzaj-Znajomości}
\hline
\underline{Id} & int \\
\hline
Nazwa & text \\
\end{attributesTable}

\begin{attributesTable}{Historia-Edukacji}
\hline
\underline{Id} & int \\
\hline
Od-kiedy & datetime \\
\hline
Do-kiedy & datetime \\
\hline
Szkoła-Id & int \\
\end{attributesTable}

\begin{attributesTable}{Szkoła}
\hline
\underline{Id} & int \\
\hline
Nazwa & text \\
\hline
Miasto-Id & int\\
\end{attributesTable}

\begin{attributesTable}{Hobby}
\hline
\underline{Id} & int \\
\hline
Nazwa & text \\
\end{attributesTable}


\begin{attributesTable}{Lista-Umiejętności}
\hline
\underline{Pracownik-Id} & int \\
\hline
\underline{Umiejętność-Id} & int \\
\end{attributesTable}


\begin{attributesTable}{Lista-Historii-Zatrudnienia}
\hline
\underline{Pracownik-Id} & int \\
\hline
\underline{Historia-Zatrudnienia-Id} & int \\
\end{attributesTable}

\begin{attributesTable}{Lista-Uprawnień}
\hline
\underline{Pracownik-Id} & int \\
\hline
\underline{Uprawnienie-Id} & int \\
\end{attributesTable}

\begin{attributesTable}{Lista-Certyfikatów}
\hline
\underline{Pracownik-Id} & int \\
\hline
\underline{Certyfikat-Id} & int \\
\end{attributesTable}

\begin{attributesTable}{Lista-Szkoleń}
\hline
\underline{Pracownik-Id} & int \\
\hline
\underline{Szkolenie-Id} & int \\
\end{attributesTable}

\begin{attributesTable}{Lista-Języków-Obcych}
\hline
\underline{Pracownik-Id} & int \\
\hline
\underline{Język-Obcy-Id} & int \\
\end{attributesTable}

\begin{attributesTable}{Lista-Historii-Edukacji}
\hline
\underline{Pracownik-Id} & int \\
\hline
\underline{Historia-Edukacji-Id} & int \\
\end{attributesTable}

\begin{attributesTable}{Lista-Hobby}
\hline
\underline{Pracownik-Id} & int \\
\hline
\underline{Hobby-Id} & int \\
\end{attributesTable}

\subsection{Diagram modelu SERM}
\includegraphics[width=\textwidth]{tex/srem.png}

\section{Projekt mechanizmów programowania serwera}

\subsection{Funkcje}
\begin{functionsTable}
	\hline	
	Znajdź-element-po-kolumnie&	
	
	Wyszukuje element w tablicy porównując wartości w podanej kolumnie&
	
	tabela:string,\newline
    kolumna:string,\newline
	wartość &	
	
	result = SELECT id \newline
	FROM tabela WHERE\newline
	kolumna = wartość\newline \newline
	return result &

	\textbf{Integer}
	
	\begin{list}{\labelitemi}{\leftmargin=1em}
	\item Id jeżeli taki element istnieje
	\item 0 jeżeli taki element nie istnieje
	\end{itemize}
	\\	

	\hline	
	Czy-Id-jest-użyty&	
	
	Zlicza ilość wystąpień podanego Id w tabeli. Jeżeli id występuje w tabeli funkcja zwraca 
	\texttt{true} w przeciwnym wypadku zwraca \texttt{false} &
		
	tabela:string,\newline
    id:int,\newline
	kolumna:string &	
	
	result = SELECT\newline
	COUNT(*) FROM table \newline
	WHERE kolumna = id\newline
	if(result != 0)\newline
		return true\newline
	else\newline
		return false\newline
	&
	
	\textbf{Boolean}
	\begin{list}{\labelitemi}{\leftmargin=1em}
	\item true są wystąpienia
	\item false nie ma wystąpień
	\end{itemize}
	\\	
	\hline
	Wartość-Słownikowa&
	
	Zwraca Id rekordu w tabeli słownikowej. Zwraca numer id lub zero w przypadku błędu &
	
	tabela:string, \newline
	wartość:string, \newline
	
	&
	
	return Znajdź-element-po-kolumnie\newline
	(tabela,Nazwa,wartość)
	
	&
	\begin{list}{\labelitemi}{\leftmargin=1em}
	\item Id jeżeli taki element istnieje
	\item 0 jeżeli taki element nie istnieje
	\end{itemize}
	\\
	
	\hline
	Dodaj-Pracownika&
	
	Dodaje kolejny rekord w tabeli pracownika. Zwraca numer id lub zero w przypadku błędu &
	
	imie:string, \newline
	nazwisko:string, \newline
	
	&
	
	return  INSERT INTO \newline
	Pracownik (‘imię’, ‘nazwisko’) \newline
	Values (imie, nazwisko);\newline
	
	&
	\begin{list}{\labelitemi}{\leftmargin=1em}
	\item Id jeżeli taki element istnieje
	\item 0 jeżeli taki element nie istnieje
	\end{itemize}
	\\	
	
	\hline
	Dodaj-Umowę&
	
	Funkcja dodaje nową umowę, zwraca id lub zero w przypadku błędu. &
	
	Pracownik-Id :int
Nazwa-typ-umowy :string
Nazwa-dzial :string:
Nazwa-stanowiska :string

	
	&
	typ-umowy-id = Wartość-Słownikowa(Typ-umowy, nazwa-typ-umowy)\newline
if(typ-umowy-id == 0 )\newline
{\newline
Return 0;\newline
}\newline

\newline
dzial-id = Wartość-Słownikowa(dzial, nazwa-dzial)\newline
if(dzial-id == 0 )\newline
{\newline
Return 0;\newline
}\newline


stanowisko-id =  Wartość-Słownikowa(Stanowisko, nazwa-stanowiska)\newline
if(stanowisko-id == 0 )\newline
{\newline
Return 0;\newline
}\newline

Return INSERT INTO Umowa \newline
 (Pracownik-Id, Typ-Umowy, Dzial-Id, Stanowisko-Id) \newline
 Values (pracownik-id, typ-umowy-id, dzial-id, stanowisko-id) \newline

	
	&
	\begin{list}{\labelitemi}{\leftmargin=1em}
	\item Id jeżeli taki element istnieje
	\item 0 jeżeli taki element nie istnieje
	\end{itemize}
	\\	

\end{relationsTable}


\subsection{Procedury}
\begin{proceduresTable}
\hline
Usun-Z-Slownika&

Usuwa wartość z słownika&

tabela : string\newline
wartość :string&

rekord-id = Wartość-Słownikowa(tabela, wartość); \newline

if(rekord-id != 0)\newline
\{\newline
	DELETE FROM tabela WHERE Id = rekord-id
\}\newline
\\

\hline
Dodaj-Miasto&
Dodaje miasto do słownika, oraz państwo jeżeli takie nie istnieje&

nazwa-miasta : string\newline
nazwa-państwa: string\newline
&

	państwo-id = Wartość-Słownikowa(Państwo, nazwa-państwa); \newline

	if(państwo-id == 0)\newline
	\{ \newline
		państwo-id = Dodaj-Do-Słownika(Państwo, nazwa-państwa);\newline
	\} \newline
	\newline
	INSERT INTO \newline
	Pracownik (‘nazwa’, ‘Państwo\_id’) \newline
	Values (nazwa-miasta, panstwo-id);\newline	
	\\
	
\hline
Dodaj-Dział&
Dodaje dział do słownika, oraz Oddział jeżeli takie nie istnieje&

nazwa-działu : string\newline
nazwa-oddziału: string\newline
&

	oddział-id = Wartość-Słownikowa(Oddział, nazwa-oddziału); \newline

	if(oddział-id == 0)\newline
	\{ \newline
		oddział-id = Dodaj-Do-Słownika(Oddział, nazwa-oddziału);\newline
	\} \newline
	\newline
	INSERT INTO \newline
	Pracownik (‘nazwa’, ‘Oddział\_id’) \newline
	Values (nazwa-działu, oddział-id);\newline	
	\\
	
\hline
Usuń-Dział&
Usuwa dział z słownika, oraz oddział jeżeli to jest ostatnie wystąpienie&

nazwa-działu : string\newline
&

	dział-id =\newline
	 Wartość-Słownikowa(Dział, nazwa-działu); \newline

	oddział-id = SELECT oddział\_id FROM dział WHERE id =\newline
	dział-id\newline

	DELETE FROM dział WHERE id =\newline
	dział-id \newline

	if(!Czy-Id-jest-użyty(Dział,oddział-id,'oddział\_id'))\newline
	\{ \newline
		DELETE FROM oddział WHERE id =\newline
		oddział-id \newline
	\} \newline
	
	\\	
	
	\hline
Usuń-Miasto&
Usuwa miasto z słownika, oraz państwo jeżeli to jest ostatnie wystąpienie&

nazwa-miasta : string\newline
&

	miasto-id =\newline
	 Wartość-Słownikowa(Dział, nazwa-miasta); \newline

	państwo-id = SELECT państwo\_id FROM miasto WHERE id =\newline
	miasto-id\newline

	DELETE FROM miasto WHERE id =\newline
	miasto-id \newline

	if(!Czy-Id-jest-użyty(Miasto,państwo-id,'państwo\_id'))\newline
	\{ \newline
		DELETE FROM państwo WHERE id =\newline
		państwo-id \newline
	\} \newline
	
	\\	
\end{proceduresTable}

\subsection{Wyzwalacze}
\begin{triggersTable}
\hline

Sprawdź-trwający-urlop	&


Sprawdza czy pracownik jest już na jakimś urlopie&
	Urlop&
		Insert&
Instead of	&
IF (NOT EXISTS \newline
		(SELECT U.Id\newline
      		FROM Urlop U,  \newline
      		     Pracownik P\newline
      		WHERE P.Id = U.Pracownik\_Id))\newline
   INSERT INTO Urlop\newline
      SELECT OD\_kiedy,\newline
      Do\_kiedy, \newline
      Pracownik-Id,Urlop-Id\newline
      FROM inserted\\
      
\hline
Sprawdź-jedna-umowy&
Sprawdza czy pracownik posiada już jakąś trwającą umowę&
Umowa&
Insert&
Instead of&
IF (NOT EXISTS \newline
		(SELECT U.Id\newline
      	FROM Umowa U,\newline
      	Pracownik P\newline
      	WHERE P.Id = U.Pracownik\_Id))\newline
{\newline
   INSERT INTO Urlop\newline
      SELECT OD\_kiedy,Do\_kiedy,\newline
       Pracownik-Id,Umowa-Typ-Id,\newline
       Stanowisko-Id\newline
      FROM inserted\newline
}\\


\hline
Dodaj-Dział&
Zapewnia poprawne wprowadzenie danych do bazy danych&
Odział&
Insert&
Instead of&

Odblokuj(Oddział) \newline
Odblokuj(Dział)\newline

Dodaj-Dział(Inserted.nazwa-oddziału,\newline
 inserted.nazwa-działu);\newline

Blokuj(Dział)\newline 
Blokuj(Oddział)\newline

\\




\hline
Dodaj-Miasto&
Zapewnia poprawne wprowadzenie danych do bazy danych&
Odział&
Insert&
Instead of&

Odblokuj(Państwo) \newline
Odblokuj(Miasto)\newline

Dodaj-Miasto(Inserted.nazwa-Państwa,\newline
 inserted.nazwa-miasta);\newline
 
Blokuj(Miasto)\newline
Blokuj(Państwo)\newline
\\

\hline
Usun-nieuzywane-oddział&
Usuwa nieużywane oddział z słownika jeżeli nie ma zapisanego działu w danym oddziale&
Miasto&
Delete&
Instead of&
Usun-Dział(deleted.Nazwa);\newline
\\

\hline
Usun-nieuzywane-państwo&
Usuwa nieużywane państwo z słownika jeżeli nie ma zapisanego miasta w danym państwie&
Miasto&
Delete&
Instead of&
Usun-Miasto(deleted.Nazwa);\newline
\\

\end{triggersTable}
\end{document}


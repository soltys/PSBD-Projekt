\begin{proceduresTable}
\hline
Usun-Z-Slownika&

Usuwa wartość z słownika&

tabela : string\newline
wartość :string&

rekord-id = Wartość-Słownikowa(tabela, wartość); \newline

if(rekord-id != 0)\newline
\{\newline
	DELETE FROM tabela WHERE Id = rekord-id
\}\newline
\\

\hline
Dodaj-Miasto&
Dodaje miasto do słownika, oraz państwo jeżeli takie nie istnieje&

nazwa-miasta : string\newline
nazwa-państwa: string\newline
&

	państwo-id = Wartość-Słownikowa(Państwo, nazwa-państwa); \newline

	if(państwo-id == 0)\newline
	\{ \newline
		państwo-id = Dodaj-Do-Słownika(Państwo, nazwa-państwa);\newline
	\} \newline
	\newline
	INSERT INTO \newline
	Pracownik (‘nazwa’, ‘Państwo\_id’) \newline
	Values (nazwa-miasta, panstwo-id);\newline	
	\\
	
\hline
Dodaj-Dział&
Dodaje dział do słownika, oraz Oddział jeżeli takie nie istnieje&

nazwa-działu : string\newline
nazwa-oddziału: string\newline
&

	oddział-id = Wartość-Słownikowa(Oddział, nazwa-oddziału); \newline

	if(oddział-id == 0)\newline
	\{ \newline
		oddział-id = Dodaj-Do-Słownika(Oddział, nazwa-oddziału);\newline
	\} \newline
	\newline
	INSERT INTO \newline
	Pracownik (‘nazwa’, ‘Oddział\_id’) \newline
	Values (nazwa-działu, oddział-id);\newline	
	\\
	
\hline
Usuń-Dział&
Usuwa dział z słownika, oraz oddział jeżeli to jest ostatnie wystąpienie&

nazwa-działu : string\newline
&

	dział-id =\newline
	 Wartość-Słownikowa(Dział, nazwa-działu); \newline

	oddział-id = SELECT oddział\_id FROM dział WHERE id =\newline
	dział-id\newline

	DELETE FROM dział WHERE id =\newline
	dział-id \newline

	if(!Czy-Id-jest-użyty(Dział,oddział-id,'oddział\_id'))\newline
	\{ \newline
		DELETE FROM oddział WHERE id =\newline
		oddział-id \newline
	\} \newline
	
	\\	
	
	\hline
Usuń-Miasto&
Usuwa miasto z słownika, oraz państwo jeżeli to jest ostatnie wystąpienie&

nazwa-miasta : string\newline
&

	miasto-id =\newline
	 Wartość-Słownikowa(Dział, nazwa-miasta); \newline

	państwo-id = SELECT państwo\_id FROM miasto WHERE id =\newline
	miasto-id\newline

	DELETE FROM miasto WHERE id =\newline
	miasto-id \newline

	if(!Czy-Id-jest-użyty(Miasto,państwo-id,'państwo\_id'))\newline
	\{ \newline
		DELETE FROM państwo WHERE id =\newline
		państwo-id \newline
	\} \newline
	
	\\	
\end{proceduresTable}
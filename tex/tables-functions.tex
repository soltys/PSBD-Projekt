\begin{functionsTable}
	\hline	
	Znajdź-element-po-kolumnie&	
	
	Wyszukuje element w tablicy porównując wartości w podanej kolumnie&
	
	tabela : string,\newline
    kolumna : string,\newline
	wartość &	
	
	result = SELECT id \newline
	FROM tabela WHERE\newline
	kolumna = wartość\newline \newline
	return result &

	\textbf{Integer}
	
	\begin{list}{\labelitemi}{\leftmargin=1em}
	\item Id jeżeli taki element istnieje
	\item 0 jeżeli taki element nie istnieje
	\end{itemize}
	\\	

	\hline	
	Czy-Id-jest-użyty&	
	
	Zlicza ilość wystąpień podanego Id w tabeli. Jeżeli id występuje w tabeli funkcja zwraca 
	\texttt{true} w przeciwnym wypadku zwraca \texttt{false} &
		
	tabela : string,\newline
    id : int,\newline
	kolumna : string &	
	
	result = SELECT\newline
	COUNT(*) FROM table \newline
	WHERE kolumna = id\newline
	if(result != 0)\newline
		return true\newline
	else\newline
		return false\newline
	&
	
	\textbf{Boolean}
	\begin{list}{\labelitemi}{\leftmargin=1em}
	\item true są wystąpienia
	\item false nie ma wystąpień
	\end{itemize}
	\\	
	\hline
	Wartość-Słownikowa&
	
	Zwraca Id rekordu w tabeli słownikowej. Zwraca numer id lub zero w przypadku błędu &
	
	tabela : string, \newline
	wartość : string, \newline
	
	&
	
	return Znajdź-element-po-kolumnie\newline
	(tabela,Nazwa,wartość)
	
	&
		\textbf{Integer}
	\begin{list}{\labelitemi}{\leftmargin=1em}
	\item Id jeżeli taki element istnieje
	\item 0 jeżeli taki element nie istnieje
	\end{itemize}
	\\
	
	\hline
	Dodaj-Do-Słownika&
	
	Zwraca Id rekordu w tabeli słownikowej. Zwraca numer id lub zero w przypadku błędu &
	
	tabela : string, \newline
	wartość : string, \newline
	
	&
	
	return INSERT INTO \newline
	tabela (‘Nazwa’) \newline
	Values (wartość);\newline
	
	&
		\textbf{Integer}
\begin{list}{\labelitemi}{\leftmargin=1em}
	\item Id jeżeli taki element istnieje
	\item 0 jeżeli taki element nie istnieje
	\end{itemize}
	\\	
	
	\hline
	Dodaj-Pracownika&
	
	Dodaje kolejny rekord w tabeli pracownika. Zwraca numer id lub zero w przypadku błędu &
	
	imie : string, \newline
	nazwisko : string, \newline
	
	&
	
	return  INSERT INTO \newline
	Pracownik (‘imię’, ‘nazwisko’) \newline
	Values (imie, nazwisko);\newline
	
	&
		\textbf{Integer}
	\begin{list}{\labelitemi}{\leftmargin=1em}
	\item Id jeżeli taki element istnieje
	\item 0 jeżeli taki element nie istnieje
	\end{itemize}
	\\	
	
	\hline
	Dodaj-Umowę&
	
	Funkcja dodaje nową umowę, zwraca id lub zero w przypadku błędu. &
	
	Pracownik-Id : int
	Nazwa-typ-umowy : string
	Nazwa-dzial : string:
	Nazwa-stanowiska : string

	
	&
	typ-umowy-id = Wartość-Słownikowa(Typ-umowy, nazwa-typ-umowy)\newline
	if(typ-umowy-id == 0 )\newline
	\{\newline
	Return 0;\newline
	\}\newline

\newline
dzial-id = Wartość-Słownikowa(dzial, nazwa-dzial)\newline
if(dzial-id == 0 )\newline
\{\newline
Return 0;\newline
\}\newline


stanowisko-id =  Wartość-Słownikowa(Stanowisko, nazwa-stanowiska)\newline
if(stanowisko-id == 0 )\newline
\{\newline
Return 0;\newline
\}\newline

Return INSERT INTO Umowa \newline
 (Pracownik-Id, Typ-Umowy, Dzial-Id, Stanowisko-Id) \newline
 Values (pracownik-id, typ-umowy-id, dzial-id, stanowisko-id) \newline

	
	&
		\textbf{Integer}
	\begin{list}{\labelitemi}{\leftmargin=1em}
	\item Id jeżeli taki element istnieje
	\item 0 jeżeli taki element nie istnieje
	\end{itemize}
	\\	
\end{functionsTable}



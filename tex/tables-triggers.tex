\begin{triggersTable}
\hline

Sprawdź-trwający-urlop	&


Sprawdza czy pracownik jest już na jakimś urlopie&
	Urlop&
		Insert&
Instead of	&
IF (NOT EXISTS \newline
		(SELECT U.Id\newline
      		FROM Urlop U,  \newline
      		     Pracownik P\newline
      		WHERE P.Id = U.Pracownik\_Id))\newline
   INSERT INTO Urlop\newline
      SELECT OD\_kiedy,\newline
      Do\_kiedy, \newline
      Pracownik-Id,Urlop-Id\newline
      FROM inserted\\
      
\hline
Sprawdź-jedna-umowy&
Sprawdza czy pracownik posiada już jakąś trwającą umowę&
Umowa&
Insert&
Instead of&
IF (NOT EXISTS \newline
		(SELECT U.Id\newline
      	FROM Umowa U,\newline
      	Pracownik P\newline
      	WHERE P.Id = U.Pracownik\_Id))\newline
{\newline
   INSERT INTO Urlop\newline
      SELECT OD\_kiedy,Do\_kiedy,\newline
       Pracownik-Id,Umowa-Typ-Id,\newline
       Stanowisko-Id\newline
      FROM inserted\newline
}\\


\hline
Dodaj-Dział&
Zapewnia poprawne wprowadzenie danych do bazy danych&
Odział&
Insert&
Instead of&

Odblokuj(Oddział) \newline
Odblokuj(Dział)\newline

Dodaj-Dział(Inserted.nazwa-oddziału,\newline
 inserted.nazwa-działu);\newline

Blokuj(Dział)\newline 
Blokuj(Oddział)\newline

\\




\hline
Dodaj-Miasto&
Zapewnia poprawne wprowadzenie danych do bazy danych&
Odział&
Insert&
Instead of&

Odblokuj(Państwo) \newline
Odblokuj(Miasto)\newline

Dodaj-Miasto(Inserted.nazwa-Państwa,\newline
 inserted.nazwa-miasta);\newline
 
Blokuj(Miasto)\newline
Blokuj(Państwo)\newline
\\

\hline
Usun-nieuzywane-oddział&
Usuwa nieużywane oddział z słownika jeżeli nie ma zapisanego działu w danym oddziale&
Miasto&
Delete&
Instead of&
Usun-Dział(deleted.Nazwa);\newline
\\

\hline
Usun-nieuzywane-państwo&
Usuwa nieużywane państwo z słownika jeżeli nie ma zapisanego miasta w danym państwie&
Miasto&
Delete&
Instead of&
Usun-Miasto(deleted.Nazwa);\newline
\\

\end{triggersTable}